\section{Experimentos e resultados}

Para cada uma das redes se calculou todas suas probabilidades usando a propriedade seguinte:
	\[ P( X_i , \ldots , X_n ) = \prod_{i=1}^{n} P( X_i \mid {Pa}( X_i ) ) \]
Onde ${Pa}( X_i )$ são todas características que são pais de $X_i$ na rede bayesiana. Além disso, a probabilidade condicional pode ser calculada da seguinte forma:
	\[ P( A \mid B ) = \frac{ P( A \cap B ) }{ P( B ) }\]
Mas para calcular todas as probabilidades, se tem que evaluar cada variável $X_i$ com os valores que aparecem os dados de entrada sendo a fórmula da seguinte forma:
	\[ P( A = a \mid B = b ) = \frac{ N( A = a \cap B = b ) + \alpha_{ab} }{N( B = b ) + \alpha_b } \]
Para este trabalho os valores de $\alpha$ foram uma estimação com base nos dados calculada com a formula:
	\[ \alpha_{a} = \frac{ 1 }{ |\Omega_a|} \]
Na formula anterior, se $a$ fosse um conjunto de variáveis, o fator debaixo da fração muda para uma produtora de $\Omega_i$.
Tendo em consideração todas as fórmulas anteriores se calcularam todas as probabilidades de cada uma das redes.
\\
Por último, com a rede já treinada pelos dados de treinamento, se tinha que testar a robustez de cada uma e fazer uma comparação entre elas usando os dados de test.
\\
Para comparar as redes foi calculado o parâmetro Data Log-Likelihood com os dados de test para cada uma. Os resultados de cada test estão na Tabela~\ref{tab:loglike}.

	\begin{table}[ h ]
		\centering
		\begin{tabular}{ | c | c | c | }
			\hline
			Rede 1 & Rede 2 & Rede 3 \\ \hline
			163794.41 & 178879.85 & 180462.16 \\ \hline
		\end{tabular}
		\caption{Data Log-Likelihood para cada rede}
		\label{tab:loglike}
	\end{table}

\clearpage