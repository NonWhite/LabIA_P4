\section{Dados de entrada}
Os dados de entrada foram extraídos da página Machine Learning Repository~\cite{UCImain} e o conjunto de dados que será usado neste trabalho é Adulto~\cite{UCIadult96}. Este conjunto tem $48842$ instancias, mas algumas não tem sua data completa. Todos os dados no conjunto são de um censo no ano 1994 feito por Barry Becker para determinar se uma pessoa tem mais de 50 mil en dinheiro por ano.

\subsection{Características}
	As características do conjunto de dados na página são:
	\begin{enumerate}
		\item Age (AG): a idade da pessoa
		\item Workclass (W): tipo de empleado que a pessoa é
		\item Fnlwgt (F): Não se da descrição sobre esta característica
		\item Education (ED): Máximo nível de educação a pessoa tem
		\item Education-num (EN): Máximo nível de educação a pessoa tem em forma numérica
		\item Marital-status (M): Estado civil da pessoa
		\item Occupation (O): A ocupação da pessoa
		\item Relationship (RE): Relações de familia como esposa ou filho
		\item Race (RA): Descrição da raza da pessoa
		\item Sex (S): sexo biológico
		\item Capital-gain (CG): Ganhos de capital
		\item Capital-loss (CL): Perdas de capital
		\item Hours-per-week (H): Horas trabalhadas por semana
		\item Native-country (N): Pais de origem da pessoa
		\item Annual-income (AI): Diz se a pessoa tem mais de 50 mil por ano
	\end{enumerate}
	Das características anteriores, não será usada a terceira (Fnlwgt) por não ter suficiente informação sobre ela. Por outro lado, algumas características tem valores discretos, mas algumas como ${Age }$ e ${Capital-gain}$ são valores continuos e portanto tem que ser discretizados, o que será feito na subseção~\ref{subsec:discret}.

\subsection{Discretização de características}
\label{subsec:discret}
	As características numéricas são:
	\begin{itemize}
		\item Age (AG)
		\item Education-num (EN)
		\item Capital-gain (CF)
		\item Capital-loss (CL)
		\item Hours-per-week (H)
	\end{itemize}
	Para cada um de eles, se calculo sua mediana e foi mudado para uma variável booleana. Então para uma característica $X_i$, temos que ${Median}( X_i )$ é sua mediana e todos seus valores mudaram da forma $x_{ij} = ( 1$ se $x_{ij} > {Median}( X_i )$ e $0$ no caso contrario ).
	
\subsection{Dados de treinamento e dados de test}
	Para poder testar e comparar as redes bayesianas que serão construídas nas seguintes seções temos que dividir os dados em dois conjuntos. O primeiro será usado para treinar e calcular todas probabilidades da rede bayesiana. Enquanto o segundo conjunto será usado para testar cada uma das redes bayesianas construídas. Para este trabalho os dados serão divididos em um 65\% para treinar e um 35\% para testar as redes.

\clearpage